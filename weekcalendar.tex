\documentclass{article}

\usepackage{luacode}

% Configuration - all settings in one place
\begin{luacode*}
cfg = {
  paper_width = 60,        -- inches
  paper_height = 36,       -- inches
  margin = .8,             -- inches
  print_dates = true,      -- should dates be printed in every square?
  start = 2025,            -- year of first row
  years = 25,              -- how many years

  -- Font sizes: \tiny \scriptsize \footnotesize \small \normalsize
  -- \large \Large \LARGE \huge \Huge
  date_font = [[\small\bfseries]],   -- dates in each box
}
\end{luacode*}

\usepackage[
  paperwidth=\directlua{tex.print(cfg.paper_width)}in,
  paperheight=\directlua{tex.print(cfg.paper_height)}in,
  margin=\directlua{tex.print(cfg.margin)}in,
% Use showframe to show the margin space
% showframe
]{geometry}

\usepackage{tikz}

\setlength{\parindent}{0pt}

\begin{document}
\thispagestyle{empty}

\begin{luacode*}
local gridlib = require("gridlib")

function getWeekDates(year, week)
    -- Validate week number (1-53)
    if week < 1 or week > 53 then
        return nil, nil, "Invalid week number"
    end

    -- ISO 8601: Week 1 is the week containing the first Thursday of the year
    -- (equivalently, the week containing January 4th)
    -- Get January 4th of the year (always in week 1)
    local jan4 = os.time({year=year, month=1, day=4})

    -- Get the day of week for Jan 4 (1=Sunday, 2=Monday, ..., 7=Saturday)
    local jan4_wday = os.date("*t", jan4).wday

    -- Calculate the Monday of week 1 (the Monday on or before Jan 4)
    -- Convert to Mon=1..Sun=7 system: Mon=2->1, Tue=3->2, ..., Sun=1->7
    local jan4_weekday = (jan4_wday == 1) and 7 or (jan4_wday - 1)
    local days_to_monday = jan4_weekday - 1  -- days from Monday to Jan 4

    -- Monday of week 1
    local week1_monday = jan4 - (days_to_monday * 24 * 60 * 60)

    -- Calculate start of requested week
    local week_start = week1_monday + ((week - 1) * 7 * 24 * 60 * 60)
    local week_end = week_start + (6 * 24 * 60 * 60)

    -- Get detailed date info for start and end
    local start_date = os.date("*t", week_start)
    local end_date = os.date("*t", week_end)

    -- Format the output string
    local formatted_date
    if start_date.month == end_date.month then
        formatted_date = string.format("%s %d--%d",
            gridlib.month_names_short[start_date.month],
            start_date.day,
            end_date.day)
    else
        formatted_date = string.format("%s %d--%s %d",
            gridlib.month_names_short[start_date.month],
            start_date.day,
            gridlib.month_names_short[end_date.month],
            end_date.day)
    end

    -- Experiment: just show the start date
    -- formatted_date = string.format("%s %d",
    --   gridlib.month_names_short[start_date.month],
    --   start_date.day)

    return formatted_date
end

function draw_grid()
  local c = cfg -- shorthand

  -- Generate labels
  local top_labels = gridlib.number_range(1, 52)
  local left_labels = gridlib.number_range(c.start, c.years)

  -- Cell content function for dates
  local function cell_content(row, col)
    if c.print_dates then
      local year = c.start + row - 1
      return string.format([[{%s %s}]], c.date_font, getWeekDates(year, col))
    end
    return ""
  end

  -- Get grid dimensions from TeX (set by geometry package)
  local paper_width, paper_height, grid_width, grid_height = gridlib.get_page_dimensions()

  gridlib.draw_grid_rowbox(grid_width, grid_height, top_labels, left_labels, {
    cell_content = cell_content,
    row_separator_interval = 5,
    row_separator_start = 5,  -- separator above 2030, 2035, 2040, ...
  })

  -- Print the pdfcrop command for test printing
  gridlib.print_crop_command("weekcalendar.pdf", paper_width, paper_height)
end
\end{luacode*}

\directlua{draw_grid()}

\end{document}
