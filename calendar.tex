\documentclass{article}

\usepackage[paperwidth=60in, paperheight=36in, margin=1in,
% Use showframe to show the margin space
% showframe
]{geometry}
\usepackage{luacode}

\usepackage{scalefnt}
\usepackage{graphicx}

\setlength{\parindent}{0pt}

\begin{document}
\thispagestyle{empty}

\begin{luacode*}
local gridlib = require("gridlib")

function getWeekDates(year, week)
    -- Validate week number (1-53)
    if week < 1 or week > 53 then
        return nil, nil, "Invalid week number"
    end

    -- Get January 1st of the year
    local jan1 = os.time({year=year, month=1, day=1})

    -- Get the day of week (1-7, where 1 is Sunday)
    local jan1_wday = os.date("*t", jan1).wday

    -- Calculate days to first Monday of year
    local days_to_monday = 0
    if jan1_wday <= 4 then
        days_to_monday = -(jan1_wday - 2)  -- 2 is Monday
    else
        days_to_monday = (9 - jan1_wday)
    end

    -- Calculate start of requested week
    local week_start = jan1 + ((days_to_monday + (week - 1) * 7) * 24 * 60 * 60)
    local week_end = week_start + (6 * 24 * 60 * 60)

    -- Get detailed date info for start and end
    local start_date = os.date("*t", week_start)
    local end_date = os.date("*t", week_end)

    -- Format the output string
    local formatted_date
    if start_date.month == end_date.month then
        formatted_date = string.format("%s %d-%d",
            gridlib.month_names[start_date.month],
            start_date.day,
            end_date.day)
    else
        formatted_date = string.format("%s %d-%s %d",
            gridlib.month_names[start_date.month],
            start_date.day,
            gridlib.month_names[end_date.month],
            end_date.day)
    end

    return formatted_date
end


-- Configuration (set these values)
cfg = {
  print_dates = true,      -- should dates be printed in every square?
  year_one_line = false,   -- should the year in the left column be on one line?
  year_count = false,      -- print year count after year?
  start = 2026,            -- year of first row
  years = 30,              -- how many years
  -- Font sizes: \tiny \scriptsize \footnotesize \small \normalsize
  -- \large \Large \LARGE \huge \Huge
  week_num_size = [[\LARGE]],      -- week numbers across the top
  year_size = [[\LARGE]],          -- year numbers in left column
  date_size = [[\footnotesize]],   -- dates in each box
}

function draw_grid()
  local c = cfg -- shorthand

  local year_margin = 0.5
  if c.year_one_line then
    year_margin = 0.8
  end

  -- Generate labels
  local top_labels = gridlib.number_range(1, 52)
  local left_labels = gridlib.number_range(c.start, c.years + 1)

  -- Custom left label function for year_count mode
  local function make_left_label(row)
    local year = c.start + row - 1
    if c.year_count then
      if c.year_one_line then
        return string.format("%d: %d", year, row - 1)
      else
        return string.format("\\vbox{%s \\hbox to %.5fin{\\hfil %d\\hfil}\\hbox to %.5fin{\\hfil %d\\hfil}}",
          c.year_size, year_margin, year, year_margin, row - 1)
      end
    else
      return tostring(year)
    end
  end

  -- Rebuild left_labels with custom formatting if needed
  if c.year_count and not c.year_one_line then
    left_labels = {}
    for i = 0, c.years do
      left_labels[#left_labels + 1] = make_left_label(i + 1)
    end
  end

  -- Cell content function for dates
  local function cell_content(row, col)
    if c.print_dates then
      local year = c.start + row - 1
      return string.format([[\rlap{\smash{%s %s}}]], c.date_size, getWeekDates(year, col))
    end
    return ""
  end

  -- Get grid dimensions from TeX (set by geometry package)
  local paper_width, paper_height, grid_width, grid_height = gridlib.get_page_dimensions()

  gridlib.draw_grid(grid_width, grid_height, top_labels, left_labels, {
    top_font_size = c.week_num_size,
    left_font_size = c.year_size,
    left_margin = year_margin,
    cell_content = cell_content,
  })

  -- Print the pdfcrop command for test printing
  gridlib.print_crop_command("calendar.pdf", paper_width, paper_height)
end
\end{luacode*}

\directlua{draw_grid()}

\end{document}